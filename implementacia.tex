\chapter{Implementácia}
V tejto časti opisujeme jazyk, ktorý sme vybrali a rôzne knižnice a doplnky, pomocou
ktorých sme vyvíjali časti nášho doplnku. Takisto uvedieme dôvody výberu.
\section{Motivácia}
bude to špecifikácia softvérového diela?
\section{Blender}
Blender je bezplatný softvér s otvoreným zdrojovým kódom, určený pre prácu s objektmi v 3D priestore. Využíva OpenGL, štandard v priemysle pre renderovanie, a je kompatibilný s operačnými systémami Windows, Linux a macOS. 
Blender poskytuje širokú škálu nástrojov pre tvorbu 3D obsahu vrátane modelovania, animácie, simulácie fyzikálnych procesov, úpravy videa a efektových vizualizácií. S relatívne nízkymi nárokmi na pamäť a disk je vhodný aj pre menej výkonné počítače. Jeho rozhranie je optimalizované pomocou OpenGL pre konzistentný výkon na rôznych hardvérových platformách. Je ideálny pre jednotlivcov a malé štúdiá, ktoré využívajú jeho jednotnú pipeline. S podporou komunity \url{blender.org/community} a malým spustiteľným súborom je Blender flexibilným a výkonným nástrojom pre tvorbu 3D obsahu na rôzne platformy. \cite{Blender}
\section{Skriptovanie}
Pre skúsených užívateľov Blender umožňuje rozšírenie svojej funkcionality a vytvorenie vlastných nástrojov pomocou skriptovacieho jazyka Python. Tento jazyk je integrovaný priamo do softvéru, teda nie je potreba samostatnej inštalácie Pythonu. Okrem toho Blender obsahuje špeciálnu knižnicu bpy, ktorá slúži na vykonávanie príkazov v Blenderi.

Používateľ má k dispozícii niekoľko prednastavených prostredí. Na vytváranie skriptov slúži prostredie Scripting, ktoré obsahuje okná na písanie, úpravu textu a Python konzolu. Je možné pracovať aj v externom programovacom prostredí.\cite{BlenderAPI}
\section{Python}
Jediným jazykom, ktorý je k dispozícii na skriptovanie v Blenderi, je programovací jazyk Python. Python je interpretovaný, vysokoúrovňový a dynamický jazyk, ktorý má široké uplatnenie v rôznych odvetviach. Jeho charakteristickým rysom je používanie odsadenia, ktoré má v tomto programovacom jazyku kľúčový význam, pretože slúži ako indikátor bloku kódu. V našej aplikácii pracujeme s verziou 3.11 jazyka Python.
\subsection{Knižnice}
Pip je systém, ktorý spravuje knižnice pre jazyk Python. Je prepojený s úložiskom
pythonovských knižníc PyPI (Python Package Index). Pre jeho používanie nie je
potrebná inštalácia, keďže je obsiahnutý v súboroch, ktoré používateľ získa inštalovaním
Pythonu.
\begin{itemize}
\item sympy: Táto knižnica pre jazyk Python poskytuje nástroje na symbolické výpočty, algebraické manipulácie, riešenie rovníc a ďalšie matematické operácie potrebné pre pokročilé výpočty.
\item numpy: Táto knižnica pre jazyk Python poskytuje nástroje na manipuláciu s vektormi, maticami, poliami a ďalšími objektami potrebnými pre zložitejšie výpočty.
\item matplotlib.pyplot: Táto knižnica v jazyku Python poskytuje rozhranie na tvorbu vizualizácií a grafického zobrazenia dát. Je často využívaná na tvorbu grafov, histogramov, kontúrových máp a ďalších typov vizuálnych reprezentácií dát.
\item math: Knižnica math v jazyku Python poskytuje základné matematické funkcie a konštanty pre numerické výpočty. Obsahuje funkcie ako $\sin, \cos, \log$ a ďalšie, ako aj konštanty ako $\pi$ a $e$.
\item mathutils: Táto knižnica je súčasťou Blenderu a poskytuje množstvo užitočných matematických funkcií a nástrojov pre prácu s 3D objektami. Obsahuje funkcie na rotácie, transformácie, výpočet normálov a ďalšie operácie v 3D priestore.
\item sys: Knižnica sys v jazyku Python poskytuje prístup k niektorým systémovým špecifikáciám a funkciám. Medzi jej použitia patrí prístup k argumentom príkazového riadku, manipulácia so štandardnými vstupmi a výstupmi, manipulácia s cestami k súborom a niektoré informácie o systéme ako verzia Pythonu a platforma
\item bpy: Ide o knižnicu Blenderu, ktorá umožňuje manipuláciu s objektmi v Blenderi pomocou príkazov vytvorených vo skriptoch.
\item time: Pomocou tejto knižnice môže používateľ získať aktuálny čas v milisekundách. 
\end{itemize}

\subsection{Implementácia}
V tejto kapitole budeme opisovať postup vývoja nášho skriptu. Takisto opíšeme jeho
funkčnosť, používateľské rozhranie a postupy, ako inštalovať potrebný softvér a knižnice.
Prvým krokom bol výber programovacieho prostredia. Následne sme potrebovali
nastaviť toto prostredie tak, aby sme v ňom mohli pracovať zároveň s Blenderom a
inštalovať všetky potrebné knižnice. Po splnení týchto krokov sme začali vyvíjať samotný skript.
\subsection{Programovacie prostredie}
Na programovanie v jazyku Python sme využívali programovacie prostredie VSCode vo verzii 1.77.1. Toto prostredie sme si vybrali pre jeho minimalistický dizajn, ktorý zjednodušuje používanie a robí ho intuitívnym a rýchlym. VSCode disponuje množstvom rozšírení, z ktorých pre našu prácu najdôležitejším je rozšírenie Blender Development. Toto rozšírenie slúži na ladenie skriptov, ktoré sa spúšťajú v prostredí Blenderu, a obsahuje aj predpripravené šablóny.

Inštalácia prostredia VSCode je možná po stiahnutí inštalačného súboru z oficiálnej webovej stránky softvéru [8]. Po stiahnutí .zip súboru požadovanej verzie programu sme spustili inštalačný súbor dvojklikom.

Rozšírenie Blender Development sme následne stiahli priamo v prostredí VSCode kliknutím na ikonu Extensions v ľavej lište rozhrania a vyhľadaním názvu rozšírenia.

Pomocou tohto rozšírenia sme vykonávali ladenie programu pomocou klávesovej skratky Ctrl+Shift+P. Po stlačení tejto klávesovej kombinácie v prostredí VSCode sa v hornej lište zobrazilo menu, v ktorom sme vybrali možnosť Blender: Build and Start a následne Blender súbor, v ktorom sme chceli ladenie vykonať. Týmto spôsobom sa softvér spustí, a opätovným použitím klávesovej skratky Ctrl+Shift+P v programovacom prostredí a výberom možnosti Blender: Run Script sa začína proces ladenia programu.
